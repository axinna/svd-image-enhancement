\chapter{结论}\label{chap:zongJie}航空图像一直是社会工作和工程项目中的研究热点,本文针对雾天拍摄得到的航空图像使用多种方法进行增强处理,针对同一种对象不同算法所得出的效果图和评价指标进行对比。并介绍了直方图均衡算法、Retinex算法和DWT-SVD算法在各个方面的优缺点。

本为从奇异值分解、离散小波变换理论出发,将两者与直方图均衡结合起来,实现了航空图像自适应的功能,结合了离散小波算法和直方图均衡的优点,并削减了它们的缺点。虽然在识别远处物体轮廓方面不如直方图均衡,但是在细节保留方面却远远的强与直方图均衡,因而能够达到一个整体效果相对好的增强图像。在去雾方面的效果与SSR算法相差不大,但是SSR算法在部分图像的增强效果甚至不如原始图像,因此本文算法在自适应方面优于SSR算法。但是在DWT-SVD算法无法对所有的雾天航空图像起到很好的增强效果,关于如何提高改算法,还有待研究。
		