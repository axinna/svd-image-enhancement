\chapter{绪论}\label{chap:xuLun}

		\section{航空图像增强概述}随着航空技术及无人机技术的快速发展,航拍图像成为了现今获取地面观察数据的重要来源。航拍图像在气象观测、国土调查、疫情控制、地理测绘等领域起重大作用。由于环境污染加重以及恶劣天气的存在,社会对航空图像的质量提出了更高的要求。例如,在沙尘、雾霾等天气条件下,图像采集设备难以获得清晰图像以便后续工作的开展。因此开展对航空图像的增强话处理意义重大。%\\ \indent
		
本课题针对模糊不清、对比度低的灰度航空图像,提出了基于奇异值分解和二维离散小波变换的图像增强处理。并通过主观评价和诸如平均值、标准差、直方图的客观评价指标进行评价。.

\section{航空图像}
			\subsection{航空图像特点}航空照片是从空中飞机上获取的特定景观或地理特征的图像,或者是从地面以上给定距离指向地球的传感器的图像。 根据他们拍摄的时间以及拍摄角度的不同,可以制作一些其他方式无法具备的具有大量信息的图片。 很多时候,从不同的角度拍摄航拍照片有助于建立准确的地图。由于人类摄影无法处理的角度和距离,航拍也对电影业非常有用。 在军事方面,航拍照片用于监视敌人的线路和间谍活动,而占星家依靠它拍摄地球和相关机构的照片。使用航拍照片可以获得立体视图,可以感受到三个维度:长度、宽度和高度,而不同于其他图片只能感知二维信息。通过查看照片可以估计地面上的物体几何特征。 虽然人眼在观看某一物体时的分辨率有限,但航空相机却没有。 航空相机比人眼对更多的光波和辐射敏感; 因此航空图像可以提供各种物体之间的相对间距和距离。

任何物体都有不同的电磁波反射和辐射特征,遥感技术是利用飞机等飞行器或者人造卫星对反射和辐射特征进行采集和分析,探测、分析和识别特定对象的技术。根据采集平台分类,遥感可以分为地面遥感、航空遥感和航天遥感;根据电磁波波长分类,遥感可分为微波遥感、红外遥感和可见光近红外遥感。由于航空遥感具有机动灵活且免受云层干扰等诸多优点,因此本文中所有的实验数据都采用“航空遥感图像”(简称为“航空图像”),更具体地,本研究中所说的航空图像指在飞机上拍摄的航拍图像,航拍图像有以下优点\cite{?}:
				\begin{itemize}
					\item \emph{航空图像信息容量大,空间分辨率大。}
					\item \emph{航拍图像应用灵活。可灵活的设置图像分辨率以及采集频率等信息,另航空图像在空间以及时间上具有较强的灵活性,有利于制定合适的方案。}
					\item \emph{航空图像信息的获取方便。}
				\end{itemize}

			 但是航拍图像也有受环境信息干扰大、勘探范围有一定的局限性的缺陷。
			\subsection{图像增强的意义}图像增强作为图像处理的一个古老而重要的分支,在不断地应用需求变化面前,也在不断更新其研究目标和发展其增强处理方法技术。通常,由于场景本身所包含的动态范围、光照条件、图像捕获设备如数码相机的局限,以及摄影者本身的技术问题等多种因素影响,多数情况下,会使得拍摄的图像达不到人们预期的目标,如场景中的运动目标产生的运动模糊、由于曝光不恰当引起的场景细节损失或是弱小目标辨识不清等,都会对后期的图像前后景分割、目标识别、目标跟踪和最终的图像理解以及预测分析等带来困难。而图像增强本身的目标就是为了突出图中感兴趣的区域、降低或去除不需要的图像信息,以此来加强和获取用户觉得有用的信息,进而得到更加适合于人/机器对图像进行理解和分析处理的表现形式或是富含更多细节信息的图像的相应处理方法\cite{?}。

图像增强基本上提高了人类对图像信息的可解读性或感知能力,并为其他自动图像处理技术提供了更好的预处理。图像增强的主要目的是修改图像的属性,使其更适合于给定的任务和特定的观察者。在此过程中,图像的一个或多个属性被修改。属性的选择和修改的方式是特定于特定任务的。此外,由于观察者的特殊因素,如人类视觉系统,将在图像增强方法的选择中引入大量的主观性。有许多技术可以在不损坏数字图像的情况下增强数字图像。图像增强分为空域图像增强和频域图像增强:

在空域图像增强技术中,我们直接处理图像像素。 操纵像素值以实现期望的增强。 在频域方法中,图像首先被转换到频域,即首先计算图像的傅里叶变换。所有的增强操作都是在图像的傅里叶上执行的,然后执行逆傅里叶变换以得到最终的图像。 执行这些增强操作是为了修改图像亮度,对比度或灰度级的分布。 结果,输出图像的像素值(强度)将根据应用于输入值的变换函数进行修改。图像增强应用于图像应该被应用和分析的每个领域。 例如,医学图像分析,卫星图像分析等。

图像处理算法提供了多种方法来改进原始图像以获得视觉上可接受的图像。根据特定任务,图像内容,观察者特征和观看条件的需求来选择这些图像增强技术。点处理方法是最原始的。对于黑暗的图像,灰度级的扩展是通过使用分数指数的幂律变换完成的。对数转换对于增强图像较暗区域中的细节非常有用,但牺牲了较亮区域中较高级别值的细节。图像的直方图能提供重要信息是关于图像的对比度。直方图均衡是通过均匀重新分配灰度值来扩展对比度的变换。只有全局直方图均衡可以简单地自动完成。
		%	\subsection{国内外研究现状}王志伟\cite{?}运用基于尺度变换和幂次变换的Retinex方法研究了模糊不清的航空图像的图像增强和道路提取,并在算法中考虑了景深信息,该算法成功使模糊不清的航空图像的的完善,为后续的道路提取打下了坚定的基础。

\section{本文主要内容与结构}本文以航空图像为研究对象,分析了航空图像的特点和图像增强的一般方法和意义,就目前适用于航空图像增强领域的常用算法展开研究,并通过实验证明本文所提出的方法的有效性。本论文主要研究内容如下:

$(1)$探究航空图像相较于其他图像的优势和其在具体的应用中受何限制。进一步探究航空图像在特定情况下的缺陷,介绍一些增强算法,使这些缺陷得到进一步的消除以便进行后续处理以及观察。

$(2)$研究常用的图像增强处理算法,分析这些方法各种航天图像增强上的效果,例举其优点及其缺点,并于本文提出的方法进行比较。

本文的具体结构安排如下:

第一章,介绍本课题的研究背景和意义,分析了航空图像的特点及其图像增强的意义,并简单介绍了一下图像增强的一般算法。

第二章,介绍了常用的图像增强方法直方图均衡算法以及Retinex增强算法。探究了直方图均衡在低对比度图像上的显著增强效果以及其在在增强的同时也模糊了细节的缺陷。并阐述了基于人眼视觉特性和颜色恒常理论的Retinex增强算法在航空图像上去雾应用上的显著特点。

第三章,介绍了本文算法DWT-SVD增强算法,详细介绍了奇异值分解算法和离散小波变换在图象上的应用以及两者的组合在图像增强上的优点。

第四章,介绍图像评价的几个指标客观指标和主观指标,并将几个算法的结果图进行比较,以及结果图的客观指标进行比较。


