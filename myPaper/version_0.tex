\documentclass[12pt]{book}
				%这里是导言区

\usepackage{ctex}
\usepackage{amsmath}
\usepackage{graphicx}
\usepackage{url}
\usepackage{listings}
\usepackage{cite}
% \setmainfont{Caladea}

\lstset{
	frame=shadowbox,
	xleftmargin=2em,
	xrightmargin=2em,
	aboveskip=1em,
	numberstyle=\tiny,
	basicstyle={\ttfamily}
}

\setlength{\oddsidemargin}{0mm}
\setlength{\evensidemargin}{0mm}
\usepackage[left=25mm,right=20mm,top=25mm,bottom=20mm]{geometry}

\linespread{1.25}

\title{基于奇异值变换的航空图像增强}
\author{王佳欣}
\date{2018年5月}

\begin{document}
	\maketitle	 	
	%标题、作者、日期 按照预定的格式展现出来
	\tableofcontents
	
	\chapter{绪论}
		\section{航空图像增强概述}随着航空技术及无人机技术的快速发展,航拍图像成为了现今获取地面观察数据的重要来源。航拍图像在气象观测、国土调查、疫情控制、地理测绘等领域起重大作用。由于环境污染加重以及恶劣天气的存在,社会对航空图像的质量提出了更高的要求。例如,在沙尘、雾霾等天气条件下,图像采集设备难以获得清晰图像以便后续工作的开展。因此开展对航空图像的增强话处理意义重大。%\\ \indent
		
本课题针对模糊不清、对比度低的灰度航空图像,提出了基于奇异值分解和二维离散小波变换的图像增强处理。并通过主观评价和诸如平均值、标准差、直方图的客观评价指标进行评价。.
		\section{航空图像的特点}
			\subsection{航空图像特点}任何物体都有不同的电磁波反射和辐射特征,遥感技术是利用飞机等飞行器或者人造卫星对反射和辐射特征进行采集和分析,探测、分析和识别特定对象的技术。根据采集平台分类,遥感可以分为地面遥感、航空遥感和航天遥感;根据电磁波波长分类,遥感可分为微波遥感、红外遥感和可见光近红外遥感。由于航空遥感具有机动灵活且免受云层干扰等诸多优点,因此本文中所有的实验数据都采用“航空遥感图像”(简称为“航空图像”),更具体地,本研究中所说的航空图像指在飞机上拍摄的航拍图像,航拍图像有以下优点\cite{?}:
				\begin{itemize}
					\item \emph{航空图像信息容量大,空间分辨率大。}
					\item \emph{航拍图像应用灵活。可灵活的设置图像分辨率以及采集频率等信息,另航空图像在空间以及时间上具有较强的灵活性,有利于制定合适的方案。}
					\item \emph{航空图像信息的获取方便。}
				\end{itemize}

			 但是航拍图像也有受环境信息干扰大、勘探范围有一定的局限性的缺陷。
			\subsection{图像增强的意义}图像增强作为图像处理的一个古老而重要的分支,在不断地应用需求变化面前,也在不断更新其研究目标和发展其增强处理方法技术。通常,由于场景本身所包含的动态范围、光照条件、图像捕获设备如数码相机的局限,以及摄影者本身的技术问题等多种因素影响,多数情况下,会使得拍摄的图像达不到人们预期的目标,如场景中的运动目标产生的运动模糊、由于曝光不恰当引起的场景细节损失或是弱小目标辨识不清等,都会对后期的图像前后景分割、目标识别、目标跟踪和最终的图像理解以及预测分析等带来困难。而图像增强本身的目标就是为了突出图中感兴趣的区域、降低或去除不需要的图像信息,以此来加强和获取用户觉得有用的信息,进而得到更加适合于人/机器对图像进行理解和分析处理的表现形式或是富含更多细节信息的图像的相应处理方法\cite{?}。
			\subsection{国内外研究现状}
			
			\subsection{本文主要内容与结构}本文提出基于奇异值与二维离散小波变换的灰度航空图像增强,并通过实验证明本文所提出的方法的有效性。本文的具体结构安排如下:

第一章,绪论。介绍本课题的研究背景和意义,分析了航空图像的特点及其图像增强的意义。

第二章,
	\chapter{常用的图像增强方法}本章首先介绍直方图均衡增强算法和Retinex增强算法的基本原理和发展历程,阐述他们在航空图像增强中的应用。	
		\section{直方图均衡增强算法}直方图描述了一幅图像的绘图统计信息,是一个关于灰度的函数,如令$x$表示灰度值,则离散函数$f(x)$表示当$x$为特定灰度值时,一幅图像上灰度值为$x$的像素的数量。

直方图均衡属于图像增强技术中的空域方法,即直接对像素值施加以相应的操作以获得增强效果。直方均衡是一种利用图像的直方图像来调整图像对比度的一项技术,该技术基于将原场景图像的直方图经重新映射后得到一个接近于均匀分布的概率密度函数的新的直方图的思想,是一种简单有效的对比度拉伸技术\cite{?}。	

直方图均衡的目的是使原始图像原有的灰度分布即直方图,重新在一定的范围内“均匀”分布。而对于原始图像直方图具有的谷值和峰值,直方图均衡处理后仍具有原来的形状,并且变得平滑。以下将阐述直方图均衡的数学表达。

句子图像处理p75
		\section{Retinex增强算法}	
	\chapter{DWT-SVD增强算法}
		\section{}
		小波变换-数字图像处理p320
	\chapter{实验结果及分析}指标

	\chapter{总结}




\end{document}